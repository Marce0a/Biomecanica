\documentclass{article}
\setlength{\parskip}{5pt} % esp. entre parrafos
\setlength{\parindent}{0pt} % esp. al inicio de un parrafo
\usepackage{amsmath} % mates
\usepackage[sort&compress,numbers]{natbib} % referencias
\usepackage{url} % que las URLs se vean lindos
\usepackage[top=25mm,left=20mm,right=20mm,bottom=25mm]{geometry} % margenes
\usepackage{hyperref} % ligas de URLs
\usepackage{graphicx} % poner figuras
\usepackage[spanish]{babel} % otros idiomas
\usepackage[utf8]{inputenc}
\author{EQUIPO 2\\Marcela Ortiz
\\Diego Ortega
\\Gabriel V\'{a}zquez
\\J\'{o}se G\'{a}lvez
\\Mart\'{i}n Villanueva\\
} % author
\title{Biomecánica\\Actividad\\Ensayo} % titulo
\date{}

\begin{document} % inicia contenido

\maketitle % cabecera


\section{Biomec\'{a}nica de la mano}\label{intro} % seccion y etiqueta
Se ha dicho, probablemente cierto, que los humanos pueden gobernar el resto del mundo.
Las criaturas de la Tierra deben su existencia a sus cerebros y manos.
El pulgar es el enemigo porque la función principal de la mano es agarrar si
Desde las patas de los crustáceos hasta las manos de los monos, no hay resistencia.
Lo hace con la destreza y precisión de una mano humana. La mano también es un importante sensor de información. Determina el tamaño y la distancia y envía la información recibida a la corteza cerebral para su interpretación y evaluación.
\\
Cuando la mano está en su posición natural, los dedos están ligeramente separados entre sí y el eje del dedo pasa por el dedo medio para conectar o separar los otros dedos. en el interior
En esta posición hay paralelismo entre los ejes de los tres últimos dedos y desplazamiento entre los ejes de los tres primeros dedos. Cuando los dedos están separados, los ejes de cada dedo convergen en un solo punto.
Esto corresponde aproximadamente a los nudos del barco. en la mano, para mover
Los movimientos disipativos y adictivos no se observan en forma de movimientos corporales simétricos, sino en forma de movimientos del eje del brazo.
Pasa por el tercer metacarpiano y el dedo medio y permanece fijo; entonces estamos hablando de un movimiento de adición o separación entre los dedos. Cuando los dedos están conectados, sus ejes no son paralelos entre sí, sino que por defecto convergen en un punto más alejado de sus bordes libres, porque los dedos
Son más anchas en la base que en la punta.
\\
Al cerrar el puño, las yemas de los dedos están rectas, el pulgar redondeado y el eje del dedo se encuentra con un punto en la base del pulgar.
poco Vale la pena señalar que en este caso el eje del dedo corresponde al dedo índice y
A partir de ahí, el movimiento es cada vez más arqueado hasta llegar al dedo meñique.
Es importante que todos los dedos mantengan la misma resistencia que el pulgar y realicen un agarre eficaz para que la mano pueda realizar su función principal de agarre.
\\
Articulaciones metacarpofalángicas Las articulaciones metacarpofalángicas son de tipo condilar y permiten la flexión activa, volar, dorsal, abducción, aducción y un pequeño movimiento de rotación axial negativa. El cabezal del peine está curvado en ambas direcciones y tiene una superficie lisa.
\\
La articulación es mucho más ancha que la base de la primera falange cóncava; pero
para una mayor estabilidad, existe el fibrocartílago glenoideo, que se inserta a través de una pequeña incisión en la palma de la base de la falange. Por lo tanto, entra en contacto con los pelos del peine durante el estiramiento fibrocartílago, que aumenta la superficie articular y el movimiento de flexión.
\\
Cuando inclinas las bisagras, se desliza sobre la palma del peine, de modo que
Rango completo de movimiento. En el proceso de estabilizar la articulación y permitir el movimiento, debe haber algo de relajación en la cápsula articular y la membrana sinovial.
Son los ligamentos laterales los que evitan el movimiento lateral cuando la articulación está flexionada, se estiran cuando la articulación está extendida y se tensan cuando la articulación está flexionada
\\
El musculo flexor profundo de los dedos se encuentra en el antebrazo capaz de flexionar los dedos, se originan desde el codo y antebrazo; una lesión en estos flexores puede ser muy graves, debido a que causan la imposibilidad de flexionar el dedo, estas lesiones se pueden originar por cortes o heridas profundas tanto en la muñeca como en los dedos, tendones, nervios y arterias. El musculo extensor de los dedos se encargan de extender las tres articulaciones del dedo, se origina en el epicóndilo lateral del humero. La presencia de ese tendón tiene importancia funcional y clínico-quirúrgica, pues en casos de lesiones en el dorso de la mano que afecten a los tendones del músculo extensor de los dedos, este tendón accesorio podría establecerse como una forma alternativa en la recuperación del movimiento de la extensión del dedo anular. Los músculos interóseos se conforman de 4 músculos localizados en la cara palmar de la mano y su función es mover la primera falange en flexión, mientras que las restantes las mueven en dirección posterior para realizar la extensión del dedo, también son capaces de contraer sus fibras para que las caras laterales de los dedos se puedan juntar.
\\
Los lumbricales son cuatro músculos pequeños que se encuentran en la mano y que tienen la característica de situarse junto a los tendones que pertenecen al músculo flexor profundo de los dedos. Estos se encuentran en el plano profundo de la palma de cada mano, su función está en la flexión de la falange proximal y en el movimiento de extensión tanto de la segunda y tercera falange, este tipo de músculos es muy inusual que se lesionen en una persona común, pero por ejemplo los deportistas son muy propensos a una rotura de estos tejidos por los diferentes movimientos que realizan. Lo que se manifiesta con dolor, limitación e incapacidad para flexionar y extender los dedos.
\\
Los músculos de la eminencia hipotenar se encuentran tres músculos que actúan directamente sobre el dedo meñique. 
\\
\begin{itemize}
  \item     El oponente: actúa sobre el quinto metacarpiano imprimiendo un movimiento de flexión y rotación alrededor de su eje longitudinal de manera que su porción anterior se dirige hacia fuera en dirección al dedo pulgar.  \\
   \item	El flexor: flexiona la primera falange sobre el primer metacarpiano, al tiempo que separa al dedo meñique del eje de la mano.  \\
   \item	El aductor: tiene la misma acción que el flexor corto. Son por tanto abductores del dedo meñique con respecto al eje de la mano.  \\
\end{itemize}


La articulación trapecio metacarpiana es la articulación básica dentro de la biomecánica del pulgar, que integra la llamada columna osteoarticular de éste, compuesta por el escafoides, trapecio, primer metacarpiano y primera y segunda falanges.Está formada por la carilla articular inferior del trapecio. Esta carilla clásicamente definida como “en silla de montar” se articula con la extremidad proximal del primer metacarpiano.
\\
El primer metacarpiano, representa la primera falange de los otros dedos. Su carilla articular para el trapecio también es cóncava en un sentido y convexa en otro.
\\
En el trapecio se asientan las inserciones musculares del oponente, del flexor corto y del abductor corto del pulgar, y en ocasiones se inserta el abductor largo. 
\\
En la base del primer metacarpiano se inserta el abductor largo del pulgar, el extensor corto de éste y el primer interóseo dorsal. 
\\
Esta articulación de anclaje recíproco permite al pulgar orientarse en relación con el resto de la mano en todos los planos del espacio.
\\
Los movimientos que realiza el pulgar por la articulación trapecio metacarpiana son de:

\begin{enumerate}
  \item     Antepulsión y retropulsión.\\
   \item	Aducción y abducción.\\
\end{enumerate}

La articulación metacarpofalángica del pulgar es de tipo condíleo, que realmente realiza también movimientos de rotación axial tanto activos como pasivos, lo que le confiere una gran importancia ya que estos movimientos no son habituales en las articulaciones de estas características.
\\
Lo que diferencia la articulación metacarpofalángica del pulgar es la presencia de dos huesos sesamoideos en el espesor de la placa palmar, donde se insertan los ligamentos metacarpo glenoideos. Los ligamentos laterales permanecen en una relativa laxitud durante la extensión articular, mientras que durante la flexión se tensan con fuerza. Los movimientos de lateralidad no existen en esta articulación.
\\
La articulación interfalángica del pulgar es de tipo troclear como el resto de las articulaciones interfalángicas, y permite sólo movimientos de flexo extensión. 
\\
La flexión es muy limitada, no alcanza más que 75-80º. La extensión activa es aproximadamente de 5 a 10º; pero la hiperextensión pasiva puede ser muy evidente, llegando hasta 30º en determinadas profesiones como los escultores o alfareros que utilizan el dedo pulgar para modelar.
\\
El abductor largo del pulgar es el más anterior de todos los tendones de la tabaquera anatómica. 
\\
El extensor corto del pulgar realiza la extensión de la primera falange.
\\
El extensor largo del pulgar es el extensor de la segunda falange del pulgar sobre la primera y extensor de la primera falange sobre el primer metacarpiano.
\\
El flexor largo propio del pulgar es realmente flexor de la tercera falange sobre la primera, desempeñando un papel insustituible en este movimiento.
\\
El aductor del pulgar actúa sobre los tres huesos del dedo. Sobre el primer metacarpiano, su acción depende de la posición en que se encuentre, así es aductor cuando el primer metacarpiano está en abducción máxima. Es abductor cuando el primer metacarpiano está en aducción máxima. Es antepulsor cuando el primer metacarpiano está en retropulsión máxima. Es retropulsor cuando el primer metacarpiano está en antepulsión.
\\
La prensión es la función primordial de la mano, la cuál es cuando el dedo pulgar se opone a los demás dedos a modo de pinza potente, desde el dedo índice al meñique con igual intensidad.
\begin{figure}[h] % figura
    \centering
    \includegraphics[width=85mm]{images/Imagen1.png} % archivo
    \caption{Prnsi\'{o}n de los dedos de la mano}
\end{figure}

La oposición del pulgar resulta de la coordinación de varios movimientos como la antepulsión y aducción del primer metacarpiano, junto con la rotación axial del primer metacarpiano y de la primera falange. Gracias a este movimiento de rotación axial, el dedo pulgar partiendo de una posición inicial en extensión máxima, con la palma muy abierta, se coloca en una posición intermedia frente al dedo índice y termina en oposición máxima contactando con el dedo meñique.
\\
El dedo pulgar es el más importante de la mano gracias a su movilidad, su fuerza, y por su capacidad de oponerse a cada uno de los demás dedos por igual y a la palma de la mano. Una retracción de la primera comisura que coloque al primer metacarpiano en retro posición, incluso conservando la movilidad de las falanges, convierten al dedo pulgar en un dedo corto y particularmente poco eficaz.
\\
El dedo pulgar es el más importante de la mano gracias a su movilidad y a su fuerza, pero sobre todo por su capacidad irremplazable de oponerse a cada uno de los demás dedos por igual y a la palma de la mano. Una retracción de la primera comisura que coloque al primer metacarpiano en retro posición, incluso conservando la movilidad de las falanges, convierten al dedo pulgar en un dedo corto y poco eficaz. 
\\
Existen varias modalidades de prensión en una mano normal, que se reparten entre las modalidades de fuerza en las que los dedos mantienen los objetos contra la palma de la mano, y las modalidades de precisión realizadas por los dedos con o sin la participación de la palma de la mano. 
\\
\begin{figure}[h] % figura
    \centering
    \includegraphics[width=85mm]{images/Imagen2.png} % archivo
    \caption{Distintas prnsiones de la mano}
\end{figure}

La prensión entre los dedos se realiza por la oposición del pulpejo del dedo pulgar con la punta de los demás dedos. requiere que todos los elementos de la mano estén en perfecto estado de funcionalidad, tanto las articulaciones como los ligamentos o los tendones. Actúan en este caso el flexor largo del índice que fija la tercera falange en flexión y el flexor largo propio del pulgar que hace lo propio sobre la tercera falange del pulgar.
\\
Para la prensión sub-terminal de los dedos es importante en este caso que la sensibilidad de los pulpejos de los dedos esté conservada y desde el punto de vista motor, la acción del flexor superficial del índice que fija la segunda falange, y los músculos tenares, flexor corto, abductor corto, primer interóseo palmar y sobre todo el aductor, que flexionan la primera falange del pulgar. La articulación interfalángica distal puede estar en extensión o incluso en semiflexión fija por una artrodesis.
\\
La prensión de subtérmino lateral de los dedos (C) es mucho más fuerte que las dos anteriores ya que involucra el pulgar y el dedo índice de una manera en la cual se tiene una mayor presión por el tipo de agarre que se hace con estos dedos.
\\
La prensión digito palmar completa (D) involucra a todos los dedos de la mano incluyendo la palma de esta, con lo que se tiene la mayor fuerza por excelencia de estas prensiones en la cual mientras se tenga una menor distancia del pulgar hacia los demás dedos de la mano se puede obtener una mayor fuerza de esta. Así que mientras más grande el objeto que se tenga en la mano será más difícil de aplicar una fuerza sobre este.
\\
La prensión digito palmar incompleta (E) también se utilizan la mayoría de los dedos junto con la palma a excepción del dedo pulgar teniendo así una gran fuerza peri no tanta como la anterior en donde los objetos se pueden llegar a resbalar hacia la palma de la mano debido a que no se utiliza el pulgar. Existen muchas maneras en las que esta prensión la utilizamos día a día como el jalar de palancas o sujetar objetos como remos.
\\
La prensión laterolateral de los dedos (F) no ejerce una gran fuerza ni precisión ya que consta del movimiento del dedo índice y el del medio para sostener objetos livianos y su función es más que todo para jugar con lápices y agarrar cigarrillos. Esta prensión se potencializa cuando a algunas personas pierden el dedo pulgar y deben de realizar maniobras sin este dedo.
\\
Para un mejor manejo de la mano es muy indispensable el contar con todos los dedos y saber aplicar la fuerza correcta y la precisión ideal para poder sostener los distintos objetos y las forma de utilizarlos de la manera más correcta con los distintos tipos de prensión que se vieron, ya que existen diferentes tareas para cada objeto al igual que a pesar de utilizar la misma maniobra, estos no estarán hechos del mismo material por lo que algunos serán más frágiles que otros.

\end{document}