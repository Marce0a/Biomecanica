
\documentclass[a4paper]{article}
\usepackage[spanish]{babel}
\selectlanguage{spanish}
\usepackage[utf8]{inputenc}
\usepackage[T1]{fontenc}

\usepackage[a4paper,top=3cm,bottom=2cm,left=3cm,right=3cm,marginparwidth=1.75cm]{geometry}

\usepackage{amsmath, amsthm, amsfonts}
\usepackage{graphicx}
\usepackage[colorinlistoftodos]{todonotes}
\usepackage[colorlinks=true, allcolors=blue]{hyperref}

%
\title{Biomecánica}
\author{Equipo 2\\
  \small Universidad Autónoma de Nuevo León\\
  \small Tarea 2\\
  \small 1849885, 1908569, 1903060, 1992076, 1870820\\
  \date{}
}

\begin{document}
\maketitle

\section{Introducci\'on}
La mano humana es una parte importante y genial de todo el cuerpo humano. Como resultado, son posibles movimientos de alta precisión con amplias posibilidades para realizar tareas, en algunos casos complejas. Actividades como tocar un instrumento musical, hacer un reloj, entrenar béisbol (en este caso no son tantas tareas, pero sí es importante la posición y el agarre de las manos y otras partes del cuerpo). Todo lo anterior proviene de la compleja fisiología de la mano, así como de un sistema avanzado de músculos y ligamentos. Puede haber prototipos y modelos de manos que puedan reproducir su complejidad, aunque también es posible hacer prótesis sin el alto nivel de sofisticación que solo el ingenio puede permitirles realizar tareas humanas básicas. 


\section{Desarrollo}
\subsection{Prótesis}
Una prótesis es un elemento desarrollado con el fin de mejorar o reemplazar una función, una parte o un miembro completo del cuerpo humano afectado, por lo tanto, una prótesis para el paciente y en particular, para el amputado, también colabora con el desarrollo psicológico del mismo, creando una percepción de totalidad al recobrar movilidad y aspecto. 

El avance en el diseño las de prótesis ha estado ligado directamente con el avance en el manejo de los materiales empleados por el hombre, así como el desarrollo tecnológico y el entendimiento de la biomecánica del cuerpo humano. La primera prótesis de miembro superior registrada data del año 2000 a. C., fue encontrada en una momia egipcia; la prótesis estaba sujeta al antebrazo por medio de un cartucho adaptado al mismo. Con el manejo del hierro, el hombre pudo construir manos más resistentes y que pudieran ser empleadas para portar objetos pesados, tal es el caso del general romano Marcus Sergius, que durante la Segunda Guerra Púnica (218-202 a. C.) fabricó una mano de hierro para él, con la cual portaba su espada, ésta es la primera mano de hierro registrada. En la búsqueda de mejoras en el año de 1400 se fabricó la mano de alt-Ruppin construida también en hierro, constaba de un pulgar rígido en oposición y dedos flexibles, los cuales eran flexionados pasivamente, éstos se podían fijar mediante un mecanismo de trinquete y además tenía una muñeca movible. 


\begin{figure}[h]
\centering
\includegraphics[width=0.2\textwidth]{protesis de alt.png}
\caption{\label{fig:biomec3}Protesis de alt- Ruppin construida a base de hierro en 1400.}
\end{figure}

No es hasta el siglo XVI, que el diseño del mecanismo de las prótesis de miembro superior se ve mejorado considerablemente, gracias al médico militar francés Ambroise Paré, quien desarrolló el primer brazo artificial móvil al nivel de codo, llamado “Le petit Loraine” el mecanismo era relativamente sencillo tomando en cuenta la época, los dedos podían abrirse o cerrarse presionando o traccionando, además de que constaba de una palanca, por medio de la cual, el brazo podía realizar la flexión o extensión a nivel de codo. Esta prótesis fue realizada para un desarticulado de codo. Paré también lanzó la primera mano estética de cuero, con lo que da un nuevo giro a la utilización de materiales para el diseño de prótesis de miembro superior. 

En el siglo XIX se emplean el cuero, los polímeros naturales y la madera en la fabricación de prótesis; los resortes contribuyen también al desarrollo de nuevos mecanismos para la fabricación de elementos de transmisión de la fuerza, para la sujeción, entre las innovaciones más importantes al diseño de las prótesis de miembro superior, se encuentra la del alemán Peter Beil. El diseño de la mano cumple con el cierre y la apertura de los dedos, pero, es controlada por los movimientos del tronco y hombro contra lateral, dando origen a las prótesis autopropulsadas. Más tarde el Conde Beafort da a conocer un brazo con flexión del codo activado al presionar una palanca contra el tórax, aprovechando también el hombro contra lateral como fuente de energía para los movimientos activos del codo y la mano. La mano constaba de un pulgar móvil utilizando un gancho dividido sagitalmente, parecido a los actuales ganchos Hook\cite{ff6}.

Toda prótesis artificial activa necesita una fuente de energía de donde tomar su fuerza; un sistema de transmisión de esta fuerza; un sistema de mando o acción y un dispositivo prensor. Por lo que podemos distribuirlos en la siguiente clasificación: 

\begin{itemize}
    \item Prótesis mecánicas:
\end{itemize}
Son dispositivos que se usan con la función de apertura o cierre voluntario por medio de un arnés el cual se sujeta alrededor de los hombros, parte del pecho y parte del brazo controlado por el usuario. Su funcionamiento se basa en la extensión de una liga por medio del arnés para su apertura o cierre, y el cierre o apertura se efectúa solo con la relajación del músculo respectivamente gracias a un resorte y tener una fuerza de presión ó pellizco.

\begin{itemize}
    \item Prótesis eléctricas:
\end{itemize}
Estas prótesis usan motores eléctricos en el dispositivo terminal, muñeca o codo con una batería recargable. Estas prótesis se controlan de varias formas, ya sea con un servocontrol, control con botón pulsador o botón con interruptor de arnés. En ciertas ocasiones se combinan estas formas para su mejor funcionalidad. Se usa un socket que es un dispositivo intermedio entre la prótesis y el muñón logrando la suspensión de éste por una succión. 

\begin{itemize}
    \item Prótesis neumáticas:
\end{itemize}
Estas prótesis eran accionadas por ácido carbónico comprimido, que proporcionaba una gran cantidad de energía, aunque también presentaba como inconveniente la complicación de sus aparatos accesorios y del riesgo del uso del ácido carbónico. 

\begin{itemize}
    \item Prótesis mioeléctricas:
\end{itemize}
Son prótesis eléctricas controladas por medio de un poder externo mioeléctrico, estas prótesis son hoy en día el tipo de miembro artificial con más alto grado de rehabilitación. Sintetizan el mejor aspecto estético, tienen gran fuerza y velocidad de prensión, así como muchas posibilidades de combinación y ampliación. 

El control mioeléctrico se basa en el concepto de que siempre que un músculo en el cuerpo se contrae o se flexiona, se produce una pequeña señal eléctrica (EMG) que es creada por la interacción química en el cuerpo. Esta señal es muy pequeña (5 a 20 µV) Un micro-voltio es una millonésima parte de un voltio. Para poner esto en perspectiva, una bombilla eléctrica típica usa 110 a 120 voltios, de forma que esta señal es un millón de veces más pequeña que la electricidad requerida para alimentar una bombilla eléctrica. 

\begin{itemize}
    \item Prótesis híbridas:
\end{itemize}
Esta combina la acción del cuerpo con el accionamiento por electricidad en una sola prótesis. En su gran mayoría, las prótesis híbridas sirven para individuos que tienen amputaciones o deficiencias transhumerales (arriba del codo) Las prótesis híbridas utilizan con frecuencia un codo accionado por el cuerpo y un dispositivo terminal controlado en forma mioeléctrica (gancho o mano)\cite{ff4}. 

\begin{figure}[h]
\centering
\includegraphics[width=0.5\textwidth]{estructura mano.png}
\caption{\label{fig:biomec1}Mano, representación de músculos, articulaciones y huesos.}
\end{figure}

\subsection{Mecanismo de la mano}
La mano puede realizar diferentes acciones como el agarre de fuerza o a mano llena, el gancho de agarre y la pinza fina o de precisión, la cual requiere contar con destrezas especiales de cada dedo comprometido en la acción. La mano humana tiene un número alto de grados de libertad, alta relación fuerza/peso, bajo factor de forma y un sistema sensorial complejo. Particularmente, cada dedo posee 2° de libertad en la base con excepción del pulgar que tiene 5° y 2 articulaciones tipo bisagra que proporcionan los movimientos de flexión y extensión.  

En la palma se encuentran otros grados de libertad, que permiten curvar la superficie donde están localizadas las bases de los dedos. El complejo sistema que constituye la funcionalidad de la mano humana, indica que cualquier alteración en ella, afecta el normal desarrollo de las actividades de una persona, y compromete su calidad de vida e incluso su autoestima. 

\begin{figure}[h]
\centering
\includegraphics[width=0.6\textwidth]{acciones mano.png}
\caption{\label{fig:biomec1}Acciones que puede hacer una mano.}
\end{figure}

\begin{itemize}
    \item Amputación del pulgar.
\end{itemize}

La funcionalidad de la mano está representada en un 40\% por la presencia del dedo pulgar. Su conservación es fundamental y en caso de tener que realizar una amputación, esta siempre debe ser lo más distal posible. Las amputaciones interfalángicas, causan la menor afectación en la funcionalidad del pulgar. En las amputaciones proximales o de todo el pulgar se debe crear un nuevo dedo con capacidad de oposición. 

\begin{itemize}
    \item Amputación de un dedo trifalángico.
\end{itemize}
Las amputaciones de los dedos se clasifican según el nivel. Las amputaciones que dejan más de la mitad de la falange proximal pueden ser funcionales, mientras que la amputación proximal a la porción media de la falange proximal es una amputación no funcional. La amputación que incluye todas las falanges y el metacarpiano del dedo comprometido proporciona una mano estéticamente más aceptable, presentándose desplazamiento de la funcionalidad del dedo faltante hacia los otros dedos. Sin embargo, el procedimiento estrecha la palma de la mano en un 20\% a 25\%, lo que reduce su capacidad para estabilizar objetos. 

\begin{itemize}
    \item Amputación pluridigitales.
\end{itemize}
En este tipo de amputación se presenta compromiso de varios dedos de la mano, que puede ser a nivel distal, interfalángico o de todo el rayo, lo que reduce la funcionalidad de la mano de acuerdo con el número de dedos comprometidos. 

\begin{itemize}
    \item Evolución de las prótesis para amputaciones parciales de la mano.
\end{itemize}
 
La investigación relacionada con el avance y el desarrollo en la biomecánica de la mano, las prótesis, los materiales biocompatibles y las prótesis cosméticas, ha venido mostrando un alto desarrollo. Los esfuerzos realizados por los investigadores en el desarrollo de nuevas prótesis de mano han generado impresionantes resultados, como la obtención de modelos que cuentan con varios grados de libertad, y diseños que se asemejan cada vez más a la mano humana y conservan su tamaño.  

Dentro de los primeros trabajos realizados en los cuales se potencian las prótesis no solo en su aspecto cosmético sino en su aspecto funcional, tanto mecánico como eléctrico, se encuentra la empresa escocesa con su producto ProDIGITS, quien provee soluciones en prótesis para distintos niveles de ausencia ya sea total o parcial. 
\begin{figure}[h]
\centering
\includegraphics[width=0.6\textwidth]{protesis antes-despues.png}
\caption{\label{fig:biomec1}Protesis ProDigits años 80 (izquierda), Protesis actual (derecha).}
\end{figure}

A mediados de los 80 se comenzó con el diseño y la fabricación de prótesis eléctricas, en la mayoría de los casos este diseño ha quedado limitado al tamaño y la edad del paciente; se ha tratado de diseñar siempre lo más pequeño posible con su propia fuente de energía y controles electrónicos adjuntados al paciente en pequeñas cajas. En 1994 un nuevo diseño fue desarrollado por Prosthetics Research Group y el Bioengineering Centre, en Edinburgh quienes resolvieron los problemas de talla y lo redujeron y simplificaron tanto en la manufactura y desarrollo de las prótesis como su control; este invento fue llamado “Prodigits” y se basó en pequeños motores y cajas de engranajes acomodados dentro de espacios digitales de los dedos y el pulgar, con lo que se resolvió el problema de espacio y se aproximó el tamaño de las prótesis al de la mano de un niño de 2 años. 

En el año 2005 se creó el primer dedo artificial diseñado específicamente para solucionar amputaciones parciales de dedo llamado “X-finger”, cada una de estas prótesis se fabrica individualmente, para acomodarla a los diferentes casos de amputación. La prótesis se recubre luego por una piel sintética de silicona, emulando casi a la perfección, la funcionalidad y la precisión de un dedo real humano no amputado.Cada una de las falanges de X-finger tiene articulaciones naturales, que se activan cuando el dedo residual se mueve. Esto permite a los usuarios comenzar a utilizar la prótesis inmediatamente, como un acto reflejo\cite{ff5}.  

\begin{figure}[h]
\centering
\includegraphics[width=0.5\textwidth]{x-finger.png}
\caption{\label{fig:biomec1}X-finger.}
\end{figure}
La gran novedad de este diseño es su composición mecánica y sencilla que permite obviar sistemas de control y motores por cada grado de libertad. Es el mejor diseño de los revisados, y por esta razón es el modelo sobre el cual se basó el diseño de la prótesis. 

\subsection{Mecanismo del dedo}
Se lleva a cabo el desarrollo e implementación de un mecanismo denominado "Dedo an­tropomórfico" el cual cumple la antropometría de la mano para un individuo (Ver figura 1). El movimiento del mecanismo se basa en la cinemática de dos subsistemas mecánicos acoplados entre sí, denominados actuadores; cada uno de ellos son mecanismos cruzados de cuatro barras. Su diseño se realiza utilizando un conjunto de diez posiciones deseadas, obtenidas de manera experimental y haciendo la interpolación gráfica y numérica de los ángulos que debe satisfacer la cinemática de cada mecanismo. La síntesis de los elementos mecánicos de los mecanismos actuadores es llevada a cabo en forma empírica y basada en el punto de vista del ingeniero de diseño con la ayuda de herramientas Computacionales de Diseño Asistido por Computadora (CAD por sus siglas en inglés). Es preciso mencionar que la metodología de diseño es desarrollada en forma particular para el conjunto de datos experimentales obtenidos previamente\cite{ff2}. 

\begin{figure}[h]
\centering
\includegraphics[width=0.5\textwidth]{prototipo UMNG.png}
\caption{\label{fig:biomec1}Prototipo de dedo realizado en la UMNG.}
\end{figure}
El propósito fundamental de obtener las relaciones matemáticas es llevar a cabo la síntesis y construcción del dedo antropomórfico por completo. Así mismo, dichas relaciones matemáticas se establecen con la finalidad de ser utilizadas para la síntesis de otros mecanismos que serán utilizados para satisfacer un nuevo conjunto de necesidades antropométricas. La síntesis dimensional de mecanismos consiste básicamente en hallar una solución a los problemas de generación de trayectoria, función y movimiento. 

\subsection{Materiales inteligentes utilizados en las prótesis de mano}
El término inteligente se utiliza como una definición para calificar y describir una serie de materiales que presentan la capacidad de cambiar sus propiedades físicas en presencia de un estímulo concreto. Para controlar dicha respuesta de una forma predeterminada, se diseñan mecanismos de control y selección. El tiempo de respuesta es corto y el sistema comienza a regresar a su estado inicial tan pronto como el estímulo cesa; la siguiente tabla nos muestra algunos de estos materiales inteligentes: 
\begin{figure}[h]
\centering
\includegraphics[width=0.6\textwidth]{materiales protesis.png}
\caption{\label{fig:biomec1}Tabla de materiales inteligentes para prótesis de mano.}
\end{figure}

Los alambres musculares, delgados y de alta resistencia mecánica, son elaborados con aleaciones de Níquel y Titanio llamadas “Nitinol", esta es una de las aleaciones con memoria más utilizadas. Uno de los aspectos críticos durante la fase de diseño de una prótesis de mano es el relacionado con la selección de los actuadores y en esta dirección los alambres musculares han mostrado una gran complementariedad con estos. 

La tendencia a futuro será incrementar la investigación y desarrollo en nuevos materiales que posean un buen comportamiento en cuanto a respuesta, compatibilidad, resistencia y durabilidad. Lo cual, junto al empleo de sistemas de control y accionamientos más potentes, robustos y compactos, posibilitará un mayor acercamiento de las prótesis de mano hacia su equivalente natural\cite{ff4}. 


\subsection{Ejemplos}

\begin{itemize}
    \item Mano Mio-eléctrica (Otto Bock).
\end{itemize}
Esta mano tiene una fuerza de agarre (100N) y una velocidad (300 mm/s), se pueden agarrar objetos rápidamente y con precisión. Se puede seleccionar un total de 6 programas diferentes con ayuda del MyoSelect 757T13 y permiten una adaptación óptima a las necesidades y capacidades del usuario de la prótesis. 

La supresión del sensor de los pulgares permite al cliente agarrar de forma activa y consciente. Los objetos se fijan y se colocan mediante señales musculares, ya que el sistema electrónico de la MyoHand no reajusta automáticamente la fuerza de agarre. Esta prótesis se recomienda a pacientes activos con un nivel de amputación bajo. Otra ventaja esencial de la mano es que el usuario puede generar de manera activa una fuerza de agarre de hasta 100 N. Gracias a los distintos programas de control puede encontrar una selección perfectamente indicada para el paciente. La velocidad y la generación de la fuerza de agarre pueden adaptarse perfectamente a las necesidades Tdel usuario mediante el MyoSelect 757T13. 
\begin{figure}[h]
\centering
\includegraphics[width=0.4\textwidth]{myoHand.png}
\caption{\label{fig:biomec1}Prótesis MyoHand}
\end{figure}

\begin{itemize}
    \item Prótesis Biónica I-Limb
\end{itemize}
La prótesis I-limb es una mano biónica cuyos dedos son controlados independientemente y por lo tanto permiten una gran cantidad de movimientos. Esta mano es capaz de hacer agarres de precisión y de potencia de diferentes formas. La mano I-limb ya ha sido implantada en pacientes de varios países. 
\begin{figure}[h]
\centering
\includegraphics[width=0.3\textwidth]{mano l-limb.png}
\caption{\label{fig:biomec1}Mano I-limb}
\end{figure}

\begin{itemize}
    \item Tercer pulgar
\end{itemize}
Dani Clode es una diseñadora británica multidisciplinar especializada en protésis. Y lleva desde el año 2017 trabajando con la idea de la mano de seis dedos. El invento se articula a través de un dispositivo situado en el pie. El movimiento del tercer pulgar se controlaba mediante sensores colocados en los dedos gordos del pie, y las comunicaciones se enviaban mediante tecnología inalámbrica colocada en la muñeca y el tobillo. Moviendo cada dedo del pie, los humanos aumentados podían mover el pulgar en diferentes direcciones y apretarlo.
\begin{figure}[h]
\centering
\includegraphics[width=0.4\textwidth]{tercer pulgar.jpg}
\caption{\label{fig:biomec1}Tercer pulgar}
\end{figure}

\section{Conclusi\'on}
La investigación que se realizo ha dejado diferentes nuevos pensamientos y conocimientos, así como nuevas posturas y diferentes perspectivas acerca del impacto de la biomecánica, ahora ya más aterrizado hacia el ambiente de las prótesis, ya que es impresionante como es que existen antecedentes en los cuales ya se practicaban el uso e implementación de estos objetos para la solución de una problemática que ha existido desde tiempos remotos. Además, de esto es impactante e interesante como es que este tema de la prótesis ha ido evolucionando a lo largo del tiempo y como es que se ha ido innovando su uso, así como la manera en que se han ido elaborando e incorporando más funcionalidades con la finalidad de atacar los diferentes puntos débiles de estos prototipos para poder brindar una mayor calidad del producto hacia la persona que hace uso del mismo.  
Lo anterior dicho se ve reflejado en cómo es que se han ido modificando los diferentes materiales con los cuales se elaboran dichas prótesis para poder brindar un mejor uso y una mejor durabilidad entre muchos otros aspectos que ayudaran a que la calidad de vida de cierta población sea mejor. 

\bibliographystyle{abbrv}
\bibliography{bib}

\end{document}
