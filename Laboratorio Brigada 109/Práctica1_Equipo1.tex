\documentclass{article}
\setlength{\parskip}{5pt} % esp. entre parrafos
\setlength{\parindent}{0pt} % esp. al inicio de un parrafo
\usepackage{amsmath} % mates
\usepackage{float}
\usepackage{blindtext}
\usepackage{subcaption}
\usepackage{wrapfig}
\usepackage[sort&compress,numbers]{natbib} % referencias
\usepackage{url} % que las URLs se vean lindos
\usepackage[top=25mm,left=20mm,right=20mm,bottom=25mm]{geometry} % margenes
\usepackage{hyperref} % ligas de URLs
\usepackage{graphicx} % poner figuras
\usepackage[spanish]{babel} % otros idiomas
\usepackage[utf8]{inputenc}
\usepackage[export]{adjustbox}
\graphicspath{{Imagenes/}}

\author{Uriel G.
Ernesto A.
Marcela O.
Ana L.
Diego O.}% author
\title{Pr\'actica 1 Descripci\'on y uso de c\'odigo de optimizaci\'on topol\'ogica} % titulo
\date{\today}

\begin{document} % inicia contenido

\maketitle % cabecera

\begin{abstract} % resumen
En esta práctica explicaremos lo que es un codigo topológico y  que sucede cuando lo optimizamos, así mismo se mostrará un ejemplo de cómo se vería un código de  optimización  topológica
\end{abstract}
%----------------------------------------------------------------------------
\section{Introducción}
La optimización topológica es una herramienta matemática que le perite a un diseñador sintetizar topologías de forma óptima. En la mecánica se entiende como topología optima a una pieza o parte mecánica diseñada especialmente para maximizar o minimizar alguna característica deseada\cite{1}. 

%----------------------------------------------------------------------------

\section{Objetivo} % seccion y etiqueta

El estudiante conocer\'a cada una de las secciones que integran el código de optimizaci\'on topol\'ogica, como se debe crear el archivo (.m) en MATLAB y como se ejecuta 
el an\'alisis.
%----------------------------------------------------------------------------

\section{Instrucciones}
El estudiante deberá presentar una propuesta de análisis de formas y de la programación, de características de trabajo específicas (programación) que presenta.
%----------------------------------------------------------------------------

\section{Estado del arte}
La optimizaci\'on topol\'ogica es una t\'ecnica englobada dentro del campo de an\'alisis estructural. Se basa en el an\'alisis mec\'anico de un componente o estructura. Su principal objetivo es el aligeramiento estructural manteniendo las funcionalidades mec\'anicas del componente objetivo. A diferencia de otros tipos de optimizaci\'on, la optimizaci\'on topol\'ogica ofrece un nuevo concepto de diseño estructural enfocado a aquellas aplicaciones donde el peso del componente es crucial\cite{2}.

Gracias a los nuevos m\'etodos computacionales, es posible llevar la optimizaci\'on a un nivel m\'as complejo de an\'alisis a nivel est\'atico, din\'amico, pl\'astico, modal o de impacto, entre otros, los cuales pueden considerarse durante el proceso de optimizaci\'on.

La optimización topol\'ogica comienza con la creaci\'on de un modelo 3D en la fase de borrador, en el que se aplicaran las diferentes cargas o fuerzas para la pieza (una presi\'on sobre las lengüetas de sujeci\'on, por ejemplo). Despu\'es, el software se encarga de calcular todas las tensiones aplicadas\cite{3}.

En este nivel, se puede realizar un corte de la pieza con el fin de retirar las partes no sometidas a las fuerzas. La geometr\'ia final, que cumple con los requisitos mec\'anicos y de diseño, se puede obtener finalmente despu\'es de alisar la pieza. De esta forma, la optimizaci\'on topol\'ogica responde a la necesidad de reducci\'on de masa además del aumento de la resistencia mec\'anica de la pieza.
%----------------------------------------------------------------------------

\section{Codigo de  programación}

\begin{figure}[h] % figura
    \centering
    \includegraphics[width=123mm]{COD1.jpeg} % archivo
    \caption{Parte 1 del codigo} 
    \label{!}
\end{figure}

\begin{figure}[P] % figura
    \centering
    \includegraphics[width=130mm]{COD2.jpeg} % archivo
    \caption{Parte 2 del codigo}   
   \label{!}
\end{figure}

%----------------------------------------------------------------------------

\section{Procedimiento de la programación}
Al empezar el programa encontramos que necesitamos colocar ciertas variables como son: •nelx, que indica el número de elementos en las direcciones 
horizontales.

•nely, que indica ese mismo valor, pero para direcciones verticales. 

•volfrac, la cual sirve para identificar la fracción de volumen. 

•penal, que hace referencia al poder de penalización. 

•rmin, esta variable dicta el tamaño del filtro el cual es dividido por el tamaño del elemento.

\begin{figure}[H] % figura
    \centering
    \includegraphics[width=130mm]{PrimeraParteDelCodigo1} % archivo
    \caption{Inicio de codigo}   
   \label{!}
\end{figure}

En esta parte se inicia por igual el análisis del material uniformemente en el dominio del diseño. Después, se inicia con la subrutina del análisis de elemento finito. Esta subrutina sirve para regresar o alojar sus resultados en un arreglo o vector de desplazamiento U.

La siguiente parte corresponde al análisis de sensibilidad y la subrutina de rigidez. En esta seccción, se realiza, usando la función “for”, un bucle de todos los elementos. Dentro de este bucle, se consigue extraer un segundo vector de desplazamiento el cual viene 
siendo el vector “Ue”. 


La segunda sección del código es el optimizador basado en criterios de optimalidad. 
Esta subrutina actualiza las variables de diseño y utiliza la sección 
“sum(sum(xnew))” la cual es una funcion monótonamente decreciente del multiplicador de Lagrange (lag) que indica el volumen del material.

\begin{figure}[H] % figura
    \centering
    \includegraphics[width=130mm]{PrimeraParteDelCodigo2} % archivo
    \caption{Sección 2 del código}   
   \label{!}
\end{figure}

La tercera seccion, consiste en un filtrado de malla, aqui se controla las variaciones en las variables en caso de que ocurra algún percance.

\begin{figure}[H] % figura
    \centering
    \includegraphics[width=130mm]{SegundaParteDelCodigo} % archivo
    \caption{Sección 3 del código}   
   \label{!}
\end{figure}

la última parte del código, corresponde al código del elemento finito y a la 
matriz de rigidez global, la cual está formada por un bucle sobre todos los elementos, algo similar a lo que se vio en una sección previa del código. Acá se vuelven a 
utilizar las variables n1 y n2 que indican los números de nodos de elementos superior izquierdo y derecho en números de nodos globales. Estos datos se obtienen y 
se usan para insertar la matriz de rigidez de elementos adecuadamente en la matrizde rigidez global. Finalmente, se realiza un calculo para determinar la matriz de rigidez del elemento. Para esto se utiliza el módulo de Young (E) y la relación de 
Poisson (nu).

\begin{figure}[H] % figura
    \centering
    \includegraphics[width=130mm]{TerceraParteDelCodigo} % archivo
    \caption{Parte final del código}   
   \label{!}
\end{figure}
\begin{figure}[H] % figura
    \centering
    \includegraphics[width=130mm]{TerceraParteDelCodigo2} % archivo
    \caption{Parte fina del código}   
   \label{!}
\end{figure}


%----------------------------------------------------------------------------
\clearpage
\section{Implementacion o desarrollo de la programación en sus diferentes vistas}
Para la primera prueba se utilizarán los valores P1(30,10,0.5,3,1.5), los resultados 
fueron los siguientes.

\begin{figure}[h]

\begin{subfigure}{0.5\textwidth}
\includegraphics[width=0.9\linewidth, height=15cm]{res1.jpeg} 
\caption{Primeros resultados}
\label{fig:subim1}
\end{subfigure}
\begin{subfigure}{0.5\textwidth}
\includegraphics[width=0.9\linewidth, height=15cm]{res2.jpeg}
\caption{Continuacion de resultados parte 2}
\label{fig:subim2}
\end{subfigure}

\caption{Lista de resultados}
\label{fig:image2}
\end{figure}

Continuacion de resultados:
\begin{figure}[H]

\begin{subfigure}{0.5\textwidth}
\includegraphics[width=0.9\linewidth, height=15cm]{Respuesta3} 
\caption{Continuacion de resultados parte 3}
\label{fig:subim1}
\end{subfigure}
\begin{subfigure}{0.5\textwidth}
\includegraphics[width=0.9\linewidth, height=8cm]{Figura}
\caption{Figura resultante }
\label{fig:subim2}
\end{subfigure}

\caption{Resultados parte 2}
\label{fig:image2}
\end{figure}

%----------------------------------------------------------------------------
\clearpage
\section{Conclusiones}
\begin{itemize}
    \item Diego O.
\end{itemize}
Ya finalizada esta práctica se puede concluir  acerca de la importancia que tiene el uso de los diferentes lenguajes de programación como lo es Matlab para poder hacer una representación mas visual de diferentes aplicaciones en la vida cotidiana, tal cual es el caso de la optimización topológica el cual al ser un análisis mecánico de una estructura tiene principal objetivo el aligeramiento estructural manteniendo las funcionalidades mecánicas del componente objetivo que ya en un caso más aterrizado a la unidad de aprendizaje seria a un objeto como una prótesis de mano, dedo entre otras.

\begin{itemize}
    \item Ana L.
\end{itemize}
En esta práctica, por medio del software MATLAB, se llevo a cabo un analisis de un codigo de optimizacion topologico, es decir, una herramienta de análisis estructural en la cual su principal objetivo es el de aligerar estructuralmente cierto componente y en donde el peso de este es crucial. En el código en MATLAB se pudieron observar varias funciones fundamentales para realizar este tipo de optimización, al igual que se aprendio a utilizar la interfaz MATLAB para hacer posible la realización de esta práctica.

\begin{itemize}
    \item Uriel G.
\end{itemize}
Al termino de realizar de la practica e investigar varios conceptos para la compresión del objetivo de la practica se puede concluir que MATLAB es una gran herramienta que ha ayudado a generar diferentes soluciones en varios campos de aplicación por medio de métodos como la optimización topológica que se basa principalmente en generar y mantener una estructura adecuada para diferentes piezas que son utilizadas en campos por mencionar la automotriz , manteniendo sus funcionalidades mecánicas . Cabe mencionar tambien que la generación del codigo fue complicada ya que algunos comandos no eran comprendidos fácilmente, por eso mismo fue necesario el realizar prueba y error para lograr solucionar el error marcado.

\begin{itemize}
    \item Marcela O.
\end{itemize}
En esta práctica utilizamos el software MATLAB enfocado al análisis de un código de optimización topológico. La programación y estudio de este caso nos ayuda en los análisis de estructuras. Para el código tuvimos imprevistos y trabas por el poco uso que hemos tenido con el software, pero pudimos corregirlo con éxito. Así mismo nos dimos cuenta que el uso de programas como MATLAB son herramientas que nos ayudan e impactan de gran manera, no solo en proyectos escolares, también su uso en la industria.

\begin{itemize}
    \item Ernesto A.
\end{itemize}
Para esta práctica realizamos una investigación para la comprensión de la optimización topologica, para ello se nos proporcionó un ejemplo el cuál pudimos comprender por medio de un código de MATLAB, el cuál pedía ciertos datos para su funcionamiento adecuado, con ello pudimos ver cómo la optimización tecnológica es usada en el campo laboral como es en el análisis de estructuras o piezas en todo caso.
\bibliographystyle{abbrv}
\bibliography{biblio}

\end{document}
