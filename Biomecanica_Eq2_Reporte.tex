
\documentclass[a4paper]{article}
\usepackage[spanish]{babel}
\selectlanguage{spanish}
\usepackage[utf8]{inputenc}
\usepackage[T1]{fontenc}

\usepackage[a4paper,top=3cm,bottom=2cm,left=3cm,right=3cm,marginparwidth=1.75cm]{geometry}

\usepackage{amsmath, amsthm, amsfonts}
\usepackage{graphicx}
\usepackage[colorinlistoftodos]{todonotes}
\usepackage[colorlinks=true, allcolors=blue]{hyperref}

%
\title{Biomecánica}
\author{Equipo 2\\
  \small Universidad Autónoma de Nuevo León\\
  \small Reporte de construcción geométrica del proyecto\\
  \small Marcela Ortiz, Diego Ortega, Gabriel Vázquez, José Gálvez, Martín Villanueva\\
  \date{}
}

\begin{document}
\maketitle

\section{Introducci\'on}
En el presente trabajo mostraremos el avance que tenemos respecto al proyecto de la construcción de una prótesis de dedo. 


\section{Diseño de la mano}

Al principio teníamos un diseño que realizamos en solidworks, pero queríamos que se viera mejor y más parecido a un dedo, que el anterior diseño. Por lo que decidimos rediseñarlo en ultimaker cura. Este software nos permite, no solo realizar el diseño de las piezas, si no, también nos da un boceto de como se puede ir imprimiendo y nos permite conectar a la impresora 3D que vayamos a utilizar.

\begin{figure}[h]
\centering
\includegraphics[width=0.5\textwidth]{disprot3.jpg}
\caption{\label{fig:biomec1}Diseño de prótesis.}
\end{figure}

\newline

\begin{figure}[h]
\centering
\includegraphics[width=0.7\textwidth]{disprot1.jpg}
\caption{\label{fig:biomec1}Diseño de prótesis.}
\end{figure}


\begin{figure}[h]
\centering
\includegraphics[width=0.7\textwidth]{disprot2.jpg}
\caption{\label{fig:biomec1}Diseño de prótesis.}
\end{figure}

Los colores de la Figura 3 (agregada en el anexo) nos indican como va imprimiéndose y como se verían los pliegues y el relieve de las piezas ya impresas. 
La parte inferior de la impresión es la base en donde se sostienen las piezas para su impresión.

\begin{figure}[H]
\centering
\includegraphics[width=0.5\textwidth]{prot1.jpg}
\caption{\label{fig:biomec1}Prótesis impresa en 3D.}
\end{figure}

\begin{figure}[H]
\centering
\includegraphics[width=0.7\textwidth]{prot2.jpg}
\caption{\label{fig:biomec1}Prótesis impresa en 3D.}
\end{figure}

\newline
\section{Conclusi\'on}
En la elaboración de la prótesis pudimos pulir nuestras habilidades de diseño, así como conocer nuevas herramientas como Ultimaker Cura, no habíamos tenido la oportunidad de trabajar e imprimir un proyecto que se imprimiera en 3D y nos gustó realizarlo. Continuaremos con el desarrollo del proyecto, con los nuevos conocimientos adquiridos hasta ahorita y los pondremos en práctica.
\newline
A continuación se anexan algunas imágenes y partes del diseño, tanto en software, como en físico.

\end{document}
